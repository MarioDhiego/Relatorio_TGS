\section{ANÁLISE DO SISTEMA}

Aplicação dos conceitos da TGS ao sistema escolhido, detalhando cada aspecto listado na metodologia.

\subsection{Objetivo do Sistema}

O TikTok tem como objetivo fundamental a criação e compartilhamento de vídeos curtos, proporcionando um espaço no qual os usuários podem se expressar de maneira dinâmica e criativa. A plataforma oferece diversas ferramentas de edição e aprimoramento de vídeos, permitindo que os conteúdos sejam personalizados e atrativos. Além desse objetivo central, o TikTok incorpora um conjunto de outros propósitos interligados, tais como:


\begin{itemize}
    \item \textbf{Entretenimento e Diversão}: O TikTok visa oferecer momentos de lazer por meio de vídeos curtos, desafios virais e tendências populares, mantendo os usuários engajados na plataforma.
    \item \textbf{Engajamento e Interação Social}: A plataforma fomenta a interação entre os usuários, criando um ambiente altamente participativo, no qual é possível curtir, comentar e compartilhar conteúdos.
    \item Expressão Criativa: Com recursos avançados de edição, trilhas sonoras, filtros e efeitos especiais, o TikTok incentiva a produção de conteúdos inovadores, permitindo que os usuários explorem sua criatividade.
    \item \textbf{Viralização e Disseminação de Conteúdo}: O algoritmo da plataforma favorece a viralização rápida dos vídeos, potencializando a propagação de desafios, trends e conteúdos de grande alcance.
    \item \textbf{Educação e Informação}: Além do entretenimento, o TikTok também se tornou um canal para a disseminação de conhecimento, oferecendo conteúdos educativos, tutoriais e informações sobre diversos temas.
\end{itemize}


Diante dessas características, observa-se que o TikTok atrai predominantemente um público jovem-adulto, impulsionado pelos seguintes fatores:


\begin{itemize}
    \item O formato de vídeos curtos facilita o consumo rápido e dinâmico de conteúdos.
    \item O algoritmo de recomendação personaliza a experiência do usuário com base em suas interações.
    \item As ferramentas de edição possibilitam a criação de conteúdos cada vez mais sofisticados e atrativos.
    \item O engajamento ocorre por meio de curtidas, compartilhamentos, comentários e reações.
     \item A cultura de desafios e trends incentiva a participação ativa e a reprodução de conteúdos.
\end{itemize}


Dessa forma, conclui-se que o TikTok não possui um único objetivo central, mas se destaca principalmente por promover o entretenimento e o engajamento, sendo esses os aspectos mais evidentes em seu funcionamento.

\newpage
\subsection{Entradas (Inputs) do Sistema}

\newpage
\subsection{Processos do Sistema}

\newpage
\subsection{Saídas (Outpus) do Sistema}


\newpage
\subsection{SubSistemas}

O algoritmo de recomendação do TikTok é um dos elementos centrais para sua popularidade global. Ele utiliza uma combinação de dados de engajamento, preferências dos usuários, informações do conteúdo e desempenho inicial para oferecer um feed personalizado. Cada interação do usuário, como curtidas, comentários e compartilhamentos, contribui para a identificação de interesses específicos. As preferências também são refinadas com base no tipo de conteúdo assistido e no tempo de visualização. Além disso, o algoritmo valoriza conteúdos originais e relevantes para tendências atuais, utilizando metadados como hashtags e descrições para categorização. Vídeos recém-publicados passam por testes iniciais com públicos pequenos antes de serem amplamente distribuídos, dependendo do engajamento gerado. A frequência de publicação e a consistência na qualidade do conteúdo também influenciam na visibilidade. Esses processos, apoiados por sistemas de inteligência artificial, garantem que os usuários recebam uma experiência única e imersiva na plataforma.

\newpage
\section{Fronteiras do Sistemas}

\newpage
\section{Retroalimentação do Sistema}

A  retroalimentação é primordial para o bom funcionamento dos sistemas, e no tiktok  por conta de seu algoritmo de recomendação existem dois tipos de retroalimentações, sendo a positiva(ocorre quando uma ação no sistema gera uma resposta que reforça ou amplifica esse comportamento, tornando-o mais frequente) e negativa (acontece quando uma ação no sistema gera uma resposta que reduz ou limita determinado comportamento), na positiva destacam-se:

\begin{itemize}
 \item \textbf{Viralização de Vídeos}: Quanto mais um vídeo recebe curtidas, comentários e compartilhamentos, mais ele é recomendado pelo algoritmo, aumentando seu alcance.
\item \textbf{Engajamento Aumentado pelo Algoritmo}: Se um usuário assiste a vídeos de um determinado tema (ex: dança, humor, tecnologia), o TikTok passa a recomendar mais conteúdos similares, incentivando o usuário a continuar assistindo.
\item \textbf{Tendências e Desafios Virais}: Quando um desafio (challenge) ou trend ganha popularidade, mais usuários participam, criando novos vídeos e reforçando ainda mais a tendência.
\item \textbf{Uso de Áudios Populares}: Músicas e sons que são utilizados em muitos vídeos aparecem com mais destaque, incentivando outros criadores a usá-los, o que aumenta ainda mais a popularidade do áudio.
\item \textbf{Interações Entre Usuários}: Se um vídeo recebe muitos comentários e respostas, o TikTok pode recomendá-lo para mais pessoas, incentivando ainda mais interações.
\item \textbf{Criadores de Conteúdo Ganhando Mais Visibilidade}: Quanto mais um criador recebe engajamento, mais seus vídeos aparecem para novos usuários, aumentando sua base de seguidores.
\item \textbf{Tempo de Permanência na Plataforma}:
O algoritmo aprende o que mantém os usuários assistindo por mais tempo e ajusta as recomendações para aumentar a retenção na plataforma.
\item \textbf{Efeito "For You" (Para Você)}:
Se um vídeo recebe muitas interações logo após ser postado, ele entra na aba "Para Você" demais usuários, aumentando ainda mais sua visibilidade.
\end{itemize}

\newpage
Com isso, a retroalimentação positiva do TikTok impulsiona a viralização de vídeos, tendências e engajamento, tornando a plataforma dinâmica e viciante para os usuários.\vskip0.3cm

Já na retroalimentação negativa no TikTok, como destacada anteriormente, ocorre quando uma ação gera uma resposta que reduz, limita ou corrige determinado comportamento, evitando excessos e equilibrando o sistema; assim, nessa retroalimentação destacam-se:

\begin{itemize}
\item \textbf{Remoção de Conteúdos Inadequados}:
Vídeos que violam as diretrizes da comunidade (como discurso de ódio, fake news ou violência) são removidos ou recebem restrição de alcance.

\item \textbf{Redução do Alcance de Vídeos Pouco Interativos}:
Se um vídeo não recebe curtidas, comentários ou compartilhamentos, o TikTok para de recomendá-lo, reduzindo sua visibilidade.

\item \textbf{Despriorização de Conteúdos Repetitivos}:
Se um usuário assiste repetidamente ao mesmo tipo de vídeo sem interagir com novos conteúdos, o algoritmo pode variar as recomendações para evitar que ele perca interesse na plataforma.

\item \textbf{Bloqueio ou Restrição de Contas}:
Contas que recebem muitas denúncias por comportamento inadequado podem ser restringidas ou banidas da plataforma.

\item \textbf{Controle de Spam}: 
Comentários considerados spam (como excesso de emojis ou links repetidos) podem ser ocultados automaticamente.

\item \textbf{Penalização de Vídeos Denunciados}:
Se um vídeo recebe muitas denúncias, ele pode ser removido automaticamente ou revisado manualmente pela equipe do TikTok.

\item \textbf{Filtragem de Conteúdos Sensíveis}:
Conteúdos considerados sensíveis (como imagens chocantes) podem ser sinalizados com avisos ou ocultados de menores de idade.
\end{itemize}



Assim, podemos analisar que ambas as retroalimentações são fundamentais para proporcionar uma experiência confortável aos usuários, garantindo um ambiente seguro e dinâmico. A retroalimentação positiva impulsiona conteúdos populares, ampliando o engajamento, enquanto a negativa filtra e limita materiais inadequados. Dessa forma, o sistema de recomendação do TikTok ajusta continuamente os conteúdos exibidos com base nas interações dos usuários, equilibrando a popularidade e a segurança da plataforma. Isso contribui para a relevância do aplicativo, tornando-o um espaço atraente e confiável para os usuários.












\section{Interdepenências}