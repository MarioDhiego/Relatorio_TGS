\section{INTRODUÇÃO}
\subsection{Contextualização}

A Teoria Geral dos Sistemas (T.G.S) ou General Theory of Systems foi proposta pelo Biólogo Autríaco Karl Ludwig Von Bertalanffy em 1937, desenvolveu seu trabalho até meados de 1948, quando se mudou para América do Norte. Em 1956 Ross Ashby introduziu o conceito na ciência cibernética. A pesquisa de Von Bertalanffy foi baseada numa visão diferente do reducionismo científico até então aplicada pela ciência convencional. Dizem alguns que foi uma reação contra o reducionismo e uma tentativa para criar a unificação científica.\vskip0.3cm

A ideia de sistema tem uma longa trajetória, remonta a Antiguidade, com pensadores como Aristóteles “o todo é maior que a soma de suas partes”, Platão e Sócrates, que já se utilizavam desse conceito à medida que procuravam formas de compreender e explicar os acontecimentos, fenômenos da natureza e o comportamento humano.\vskip0.3cm

As aplicações da Teoria Geral dos Sistemas teve início nos Estados Unidos da América (USA) com aplicações à: - Biologia - Termodinâmica.\vskip0.3cm

Alguns anos a frente sua aplicação se fez presente em outras áreas, tais como:

\begin{itemize}
    \item Ecologia (TANLEY, 1937);
    \item Geografia (SOTCHAVA, 1977);
    \item Psicologia Social das Organizações (KATZ e KAHN, 1966);
    \item Psiquiatria (GRINKER, 1967);
    \item Psicologia do Desenvolvimento (BRONFRENBRENNER, 1977);
    \item Economia (BOULDING, 1953);
    \item Administração (SCOTT, 1963);
\end{itemize}

\newpage
\subsection{Apresentação do TikTok como Objeto de Estudo}

O TikTok, lançado inicialmente como Douyin na China em setembro de 2016 pela empresa ByteDance, rapidamente se transformou em uma das plataformas de mídia social mais influentes e populares do mundo. A versão internacional do app, com o nome de TikTok, foi apresentada ao público global em setembro de 2017. A plataforma foi projetada para permitir que os usuários criassem e compartilhassem vídeos curtos, com duração de até 60 segundos, normalmente acompanhados de músicas, efeitos especiais e desafios virais.\vskip0.3cm

A ascensão meteórica do TikTok foi acelerada em 2018, quando a ByteDance adquiriu o aplicativo musical.ly, que já era bastante popular entre adolescentes no Ocidente. A fusão das duas plataformas resultou na criação de um ambiente único, onde usuários de diferentes partes do mundo podiam se conectar e interagir por meio de vídeos criativos e espontâneos.\vskip0.3cm

A proposta inovadora do TikTok, que combina algoritmos poderosos de recomendação com uma interface simples e intuitiva, foi um dos principais fatores de seu sucesso. A plataforma é alimentada por um algoritmo de inteligência artificial que personaliza a experiência de cada usuário, exibindo conteúdos que têm mais chances de gerar engajamento, o que aumenta a quantidade de tempo que as pessoas passam no aplicativo.\vskip0.3cm

Em poucos anos, o TikTok se transformou em uma verdadeira potência da mídia social, com mais de 1 bilhão de usuários ativos mensais, influenciando não apenas o comportamento dos usuários, mas também a indústria musical, a publicidade e até mesmo a cultura pop global. Ele se tornou um espaço para lançamentos de tendências, campanhas de marketing inovadoras e expressão pessoal criativa, consolidando-se como um fenômeno global, especialmente entre a geração Z.




\newpage
\subsection{Objetivo do Sistema}


Propósito principal do TikTok (oferecer uma plataforma para criação, compartilhamento, consumo de vídeos curtos e engajamento social).

\subsection{Entradas (Inputs) do Sistema}


\subsection{Processos do Sistema}

\subsection{Saídas (Outpus) do Sistema}