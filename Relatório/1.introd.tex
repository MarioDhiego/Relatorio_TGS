\section{INTRODUÇÃO}
\subsection{Contextualização sobre a Teoria Geral dos Sistemas}

A Teoria Geral dos Sistemas (T.G.S) ou General Theory of Systems foi proposta pelo Biólogo Autríaco \textbf{Karl Ludwig Von Bertalanffy} em 1940, desenvolveu seu trabalho até meados de 1948, quando se mudou para América do Norte. Em 1956 \textbf{Ross Ashby} introduziu o conceito na ciência cibernética. A pesquisa de Von Bertalanffy foi baseada numa visão diferente do reducionismo científico até então aplicada pela ciência convencional. Dizem alguns que foi uma reação contra o reducionismo e uma tentativa para criar a unificação científica (BERTALANFFY, 2008) e (COSTA,2020).\vskip0.3cm

A ideia de sistema tem uma longa trajetória, remonta a Antiguidade, com pensadores como \textbf{Aristóteles} “o todo é maior que a soma de suas partes”, \textbf{Platão} e \textbf{Sócrates}, que já se utilizavam desse conceito à medida que procuravam formas de compreender e explicar os acontecimentos, fenômenos da natureza e o comportamento humano.\vskip0.3cm

O princípio geral da teoria é que um sistema não pode ser compreendido apenas pela análise de suas partes isoladamente, pois o comportamento do todo depende das interações e das relações entre essas partes. Essa perspectiva influenciou diversas áreas, como biologia, sociologia, psicologia, administração e ciência da computação.\vskip0.3cm



As aplicações da Teoria Geral dos Sistemas teve início nos Estados Unidos da América (USA) com aplicações à: - Biologia - Termodinâmica.\vskip0.3cm

Alguns anos a frente sua aplicação se fez presente em outras áreas, tais como:

\begin{itemize}
    \item Ecologia (TANLEY, 1937);
    \item Geografia (SOTCHAVA, 1977);
    \item Psicologia Social das Organizações (KATZ e KAHN, 1966);
    \item Psiquiatria (GRINKER, 1967);
    \item Psicologia do Desenvolvimento (BRONFRENBRENNER, 1977);
    \item Economia (BOULDING, 1953);
    \item Administração (SCOTT, 1963);
\end{itemize}

\newpage
\subsection{Panorama do Marketing Digital Global}


A mídia social e o conteúdo gerado pelo usuário definem tecnologias interativas projetadas para criar e compartilhar informações, ideias e interesses entre comunidades virtuais. A mídia social varia de acordo com postagens de texto, imagens, vídeos, comentários e plataformas de rede. A taxa de penetração global atingiu 66,2\% e 5,22 bilhões de usuários de mídia social em todo o mundo em 2024. O norte e o oeste da Europa tiveram a maior taxa de penetração, seguidos pelo leste da Ásia e pelo sul da Europa. A Ásia ostenta hoje, quase três bilhões de usuários de mídia social, tornando-se o continente com maior audiência (STATISTICA, 2024).\vskip0.3cm

Segundo o Relatório de Visão Geral Global Digital da DataReportal(2024), em dezembro de 2024, as redes sociais mais populares, por números de usuários ativos mensais (em milhões) eram:


\begin{enumerate}
    \item Facebook = 3.065
    \item YouTube = 2,504
    \item WhatsApp = 2,000
    \item TikTok = 1,582
    \item WeChat = 1,343
    \item Facebook Messenger = 1,010
    \item Telegram = 900
\end{enumerate}

No Brasil, 62,3\% da população utiliza pelo menos uma rede social, indicando que ainda há espaço para crescimento. Já em relação aos aplicativos, há uma dominância do WhatsApp, utilizado por 83,2\% dos usuários no mundo (DATAREPORTAL, 2024). \vskip0.3cm

\begin{enumerate}
    \item WhatsApp = 83,2\%
    \item Facebook = 64,1\%   
    \item YouTube = 63,7\%
    \item Line = 62,9\%
    \item TikTok = 61,7\%
\end{enumerate}

\newpage

Em relação ao tempo médio mensal gasto por usuário usando redes sociais, verificou-se que, o TikTok estar no topo do ranking, cerca de 34 horas, o que representa mais de uma hora por dia. Na sequência, aparecem o YouTube (28 horas e 5 minutos) e Facebook (19 horas e 47 minutos).  \vskip0.3cm


Esses números refletem a popularidade e o alcance global dessas plataformas em 2024.





\subsection{Apresentação do TikTok como Objeto de Estudo}

O TikTok, lançado inicialmente como Douyin na China em setembro de 2016 pela empresa ByteDance, rapidamente se transformou em uma das plataformas de mídia social mais influentes e populares do mundo. A versão internacional do app, com o nome de TikTok, foi apresentada ao público global em setembro de 2017. A plataforma foi projetada para permitir que os usuários criassem e compartilhassem vídeos curtos, com duração de até 60 segundos, normalmente acompanhados de músicas, efeitos especiais e desafios virais 
 (KAYE et al, 2020).\vskip0.3cm

A ascensão meteórica do TikTok foi acelerada em 2018, quando a ByteDance adquiriu o aplicativo musical.ly, que já era bastante popular entre adolescentes no Ocidente. A fusão das duas plataformas resultou na criação de um ambiente único, onde usuários de diferentes partes do mundo podiam se conectar e interagir por meio de vídeos criativos e espontâneos.\vskip0.3cm

A proposta inovadora do TikTok, que combina algoritmos poderosos de recomendação com uma interface simples e intuitiva, foi um dos principais fatores de seu sucesso. A plataforma é alimentada por um algoritmo de inteligência artificial que personaliza a experiência de cada usuário, exibindo conteúdos que têm mais chances de gerar engajamento, o que aumenta a quantidade de tempo que as pessoas passam no aplicativo.\vskip0.3cm

Em poucos anos, o TikTok se transformou em uma verdadeira potência da mídia social, com mais de 1,582 bilhão de usuários ativos mensais, influenciando não apenas o comportamento dos usuários, mas também a indústria musical, a publicidade e até mesmo a cultura pop global. Ele se tornou um espaço para lançamentos de tendências, campanhas de marketing inovadoras e expressão pessoal criativa, consolidando-se como um fenômeno global, especialmente entre a geração Z (STATISTICA, 2024).\vskip0.3cm

\newpage
\subsection{Restrições de Uso em Países}

O TikTok, uma das redes sociais mais populares do mundo, tem enfrentado diversas restrições e bloqueios em vários países devido a questões de segurança nacional, privacidade de dados e preocupações com o conteúdo gerado pelos usuários, até questões culturais e religiosas.\vskip0.3cm

Elas podem ser classificadas em três categorias principais: bloqueios totais, restrições parciais e limitações específicas para determinados usuários ou grupos.


\subsubsection{India}

Em 2020, o TikTok foi banido indefinidamente na Índia após o país ter um confronto de fronteira com a China. O banimento ocorreu após um aumento das tensões entre Índia e China devido a um confronto militar na região de Ladakh. As autoridades indianas alegaram preocupações de segurança cibernética e proteção de dados, argumentando que esses aplicativos poderiam representar um risco para a soberania e integridade do país.
\vskip0.3cm

\begin{figure}[H]
    \centering
    \includegraphics[width=0.7\linewidth]{TIKTOK1.jpg}
    \caption{Bloqueio do TikTok na India (The Gardian, 2020)}
    \label{fig:enter-label} 
\end{figure}


\subsubsection{Afeganistão}

A liderança Talibã do Afeganistão proibiu o TikTok e o jogo PUBG em 2022, com o argumento de proteger os jovens de "serem enganados". Essa ação reflete o controle rígido que o Talibã busca impor sobre a sociedade e as ferramentas digitais desde que voltou ao poder em agosto de 2021.


\subsubsection{Taiwan}

Em dezembro de 2022, Taiwan impôs uma proibição do setor público ao TikTok depois que o FBI alertou que o TikTok representava um risco à segurança nacional.


\subsection{França}

Em 24 de março de 2023, o governo da França proibiu a instalação e o uso de aplicativos "recreativos", incluindo a rede social chinesa TikTok e a plataforma de streaming americana Netflix nos telefones de trabalho dos 2,5 milhões de funcionários públicos do país.


\subsubsection{Canadá}
Em 6 de novembro de 2024, o Canadá ordenou que o TikTok encerrasse seus escritórios no país devido a preocupações de segurança nacional, mas o acesso ao aplicativo não foi proibido. No entanto, os usuários ainda poderão acessar o aplicativo de vídeo e carregar conteúdo nele.
