%CAPA

\begin{center}
    \thispagestyle{empty} %comando que oculta o número da página
    \includegraphics[scale=0.6]{Figuras/Logo UFPA.png}
    \onehalfspacing

    UNIVERSIDADE FEDERAL DO PARÁ \\
    INSTITUTO DE CIÊNCIAS EXATAS E NATURAIS \\
    FACULDADE DE COMPUTAÇÃO\\
    CURSO DE SISTEMAS DE INFORMAÇÃO \\
    
    \vspace{4cm}
    Carlos Eduardo Maués Mendes\\
    Luiz Carlos Rodrigues Moura\\
    Rodrigo de Sousa Gois\\
    Mário Diego Rocha Valente\\
    Luiz David Amorim Fernandes\\
    \vspace{5cm}
    
    \textbf{Aplicação da Teoria Geral Dos Sistemas em Tecnologias do Cotidiano: Estudo de Caso a Rede Social \textit{TikTok}}\\
    \vspace{5cm}
    
    BELÉM\\
    2025\\
    
%FOLHA DE ROSTO

    \newpage
    \setcounter{page}{1} %comando que redefine a contagem de páginas
    \thispagestyle{empty}
    
    Carlos Eduardo Maués Mendes\\
    Luiz Carlos Rodrigues Moura\\
    Rodrigo de Sousa Gois\\
    Mário Diego Rocha Valente\\
    Luiz David Amorim Fernandes\\
    \vspace{5cm}
    
    APLICAÇÃO DA TEORIA GERAL DOS SISTEMAS AO TikTok\\
    \vspace{5cm}
    
\end{center}

\singlespacing
\hspace{8cm} % posicionando a caixa de texto
\begin{minipage}{7cm}
Relatório de Trabalho Técnico apresentado como avaliação parcial da disciplina (Teoria de Sistemas Aplicada à Informática) ministrada pelo professor Mc. (Dejan Martins Conceição) para o curso de Bacharelado em Sistemas de Formação do Instituto de Ciências Exatas e Naturais, da Universidade Federal do Pará (UFPA) - Campus Belém. 
\end{minipage}
\vspace{4cm}

\onehalfspacing
\begin{center}
    BELÉM\\
    2025
\end{center}

\newpage
\thispagestyle{empty}
\begin{center}

\begin{abstract}
    Este trabalho tem como objetivo analisar a plataforma TikTok à luz da Teoria Geral dos Sistemas (TGS). A TGS, proposta por Ludwig von Bertalanffy, visa compreender como sistemas, de diferentes naturezas, podem ser analisados através de suas interações internas e com o ambiente externo. A pesquisa busca aplicar os conceitos fundamentais da TGS, como entradas, processos, saídas, subsistemas, retroalimentação e interdependência, à plataforma TikTok. A análise considera a plataforma como um sistema aberto, destacando sua capacidade de interação com usuários e a adaptação a novas tendências. Ao final, serão discutidos os resultados da análise e sugestões para aprimoramento do sistema.

\end{abstract}

\newpage
\tableofcontents


\end{center}
