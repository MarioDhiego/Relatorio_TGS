\section{CONSIDERAÇÕES FINAIS}

A aplicação dos conceitos da Teoria Geral de Sistemas (TGS) ao TikTok permitiu uma compreensão profunda e estruturada do funcionamento dessa plataforma como um sistema complexo e dinâmico. Ao analisar seus componentes, inter-relações e interdependências, foi possível identificar como a TGS contribui para entender e aprimorar sistemas tecnológicos como o TikTok. \vskip0.3cm

Essa interdependência demonstra que o TikTok é um sistema integrado, onde a falha em um subsistema pode impactar todo o funcionamento da plataforma. Portanto, a TGS reforça a necessidade de equilíbrio e coordenação entre os componentes para manter a eficiência e a relevância do sistema. \vskip0.3cm

O estudo do TikTok sob a ótica da TGS evidenciou a complexidade e a eficiência dessa plataforma como um sistema tecnológico. A análise das entradas, processos, saídas, subsistemas, fronteiras e retroalimentação permitiu compreender não apenas seu funcionamento, mas também identificar oportunidades de melhoria. A TGS mostrou-se uma ferramenta valiosa para entender e aprimorar sistemas complexos, destacando a importância da interdependência e do equilíbrio entre seus componentes. No caso do TikTok, essa abordagem pode contribuir para uma plataforma mais transparente, segura e sustentável, alinhada às necessidades dos usuários e da sociedade.



